% Created 2020-12-26 sam. 23:21
% Intended LaTeX compiler: pdflatex
\documentclass[11pt]{article}
\usepackage[utf8]{inputenc}
\usepackage[T1]{fontenc}
\usepackage{graphicx}
\usepackage{grffile}
\usepackage{longtable}
\usepackage{wrapfig}
\usepackage{rotating}
\usepackage[normalem]{ulem}
\usepackage{amsmath}
\usepackage{textcomp}
\usepackage{amssymb}
\usepackage{capt-of}
\usepackage{hyperref}
\usepackage{tabularx}
\author{Vijay Gopal Chilkuri}
\date{\today}
\title{QMC Part I}
\hypersetup{
 pdfauthor={Vijay Gopal Chilkuri},
 pdftitle={QMC Part I},
 pdfkeywords={},
 pdfsubject={},
 pdfcreator={Emacs 27.1 (Org mode 9.5)}, 
 pdflang={English}}
\begin{document}

\maketitle
\tableofcontents


\section{Part I}
\label{sec:org99a81f6}

\section{Simple power method}
\label{sec:org29cc75c}

\subsection{2x2 Matrix}
\label{sec:org7af4f28}

Consider a Hamiltonian whose matrix form is shown in Eq:\ref{mat2x2}. This Hamiltonian
has two eigenvectors and two eigenvalues. Let the two eigenvectors be
\(\mathbf{u_0}\) and \(\mathbf{u_1}\). Then consider \(\mathbf{\nu}\) any arbitrary trial vector which is
not an eigenvector of Eq:\ref{mat2x2} but belongs to the 2x2 space. Any such trial vector
can always be written in the form shown in Eq:\ref{vecbasis}.

\begin{equation}
\begin{bmatrix}
 v_0 & -t \\
 -t & v_1 \\
\end{bmatrix}
\label{mat2x2}
\end{equation}

\begin{equation}
\label{vecbasis}
\mathbf{\nu} = c_0 \mathbf{u_0} + c_1 \mathbf{u_1}
\end{equation}

The key idea is the realization that the ground state of the Hamiltonian \ref{mat2x2}
given by \(\mathbf{u_0}\) can be extracted from \(\mathbf{\nu}\) by the repeated
application of a filter \(G(H)\). This filter systematically purifies
\(\mathbf{\nu}\) to obtain the ground state \(\mathbf{u_0}\) provided \(c_0 > 0\),
i.e. the trial vector \(\mathbf{\nu}\) has a non-zero projection on the
ground state.

The form of the filter is inspired from the power method where a successive
application of the Hamiltonian followed by the substraction of the residual
leads to a convergent series of vectors. The limiting value of this convergent
series is one of the extremal eigenvectors of the Hamiltonian. Following this, our filter \(G(H)\) can be written as \ref{filter}

\begin{equation}
\label{filter}
\hat{G}(H) = \left ( \mathbf{1} - \tau (\hat{H} - E_T\mathbf{1}) \right)
\end{equation}

The convergent series of vectors is then \(\left\{ \nu^{(0)},  \nu^{(1)},\
\nu^{(2)},\dots,\nu^{(n)}\right\}\) where \(\nu^{(k)}\) is given by \ref{applyg}

\begin{equation}
\label{applyg}
\nu^{(k+1)} = \hat{G}(H)\nu^{(k)}
\end{equation}

Using Eq:\ref{vecbasis}, Eq:\ref{applyg} can be written as \ref{applyg2}

\begin{equation}
\label{applyg2}
\nu^{(k+1)} = c_0 (1-\tau(E_0-E_T))^{(k)}\mathbf{u_0} + c_1 (1-\tau(E_1-E_T))^{(k)}\mathbf{u_1}
\end{equation}

From Eq:\ref{applyg2}, we can see that a repeated application of the filter Eq:\ref{filter}
with a trial guess energy \(E_T\) will result in the series converging
geometrically to either \(E_0\) or \(E_1\) depending on the choice of \(E_T\).

Here we shall show an example using the Hamiltonian given in Eq:\ref{mat2x2}. In the
case of a 2x2 Hamiltonian, and any general trial vector \([c_0,c_1]\), the
recursion relations for the calculation of \(c^{(k)}_0\) and \(c^{(k)}_1\) are
straight forward and given by Eq:\ref{receqn1},\ref{receqn2}.

\begin{equation}
\label{receqn1}
c^{(k+1)}_0 =  \left(\mathbf{1}-\tau\left(\nu_0 - E_T\right)\right)c^{(k)}_0 + \tau t c^{(k)}_1
\end{equation}

\begin{equation}
\label{receqn2}
c^{(k)}_1 = \tau t c^{(k)}_0 + \left(\mathbf{1}-\tau\left(\nu_1 - E_T\right)\right)c^{(k)}_1
\end{equation}

These are the working equations. As an example, we begin with the initial set of
values given as follows:

\begin{verbatim}
\nu_0=1
\nu_1=2
t=1
\tau=0.1
E_T=3
c_0=0.31622776601683794
c_1=0.9486832980505138
\end{verbatim}

The iteration can begin with these as starting values. The output is given in
shown in the Figure: .

\begin{figure}[htbp]
\centering
\includegraphics[width=15cm]{/home/vijay/Documents/pedagogical_qmc/figure1.pdf}
\caption{\label{figure1}Convergence of the Local energy as a function of iterations.}
\end{figure}
\end{document}
